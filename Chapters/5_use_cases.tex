\chapter{Cas d'utilisation}

Une fois la carte électronique développée, il a été nécessaire de penser à quels types de \textit{use cases} qui pourrait être utilisés comme démonstration complète de la plateforme.

Pour cela le but était d'utiliser l'intégralité de l'infrastructure mise à disposition. C'est à dire, de partir du \textit{device} (SmartCantonDevBox dans notre cas) jusqu'à la récupération des données depuis la base de données.

\section{Échangeur d'APPKEY LoRaWAN}

Ce premier \textit{use case} a pour but de 

Comme expliqué sous le chapitre \ref{sec_lora_security}



\section{Tacker GPS}

Ce use case est sûrement l'un des plus simple à mettre en place mais également l'un des plus visuel. Le but étant de pouvoir reporter la position d'un objet avec une intervalle de temps programmable par l'utilisateur.

Cette position est ensuite visible sur une application accédant au Context Broker. Le déplacement de 

\section{Scanneur Bluetooth de proximité}






\section{Capteur générique DevBox}

Ce use case a pour but d'utiliser tous les périphériques disponibles sur la carte électronique développée. On a donc le transfert de la position GPS, de la qualité de l'air ainsi que de l'orientation de la carte électronique (données accéléromètre et magnétomètre).

